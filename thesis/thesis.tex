\documentclass{amsart}

\usepackage{todonotes}

\usepackage{hyperref}
\usepackage[german]{babel}

\usepackage{listings}
\renewcommand{\lstlistingname}{Algorithm}

\usepackage{enumerate}

\newtheorem{theorem}{Theorem}[section]
\newtheorem{proposition}[theorem]{Proposition}
\newtheorem{lemma}[theorem]{Lemma}
\newtheorem{corollary}[theorem]{Corollary}

\theoremstyle{definition}
\newtheorem{definition}[theorem]{Definition}
\newtheorem{example}[theorem]{Example}
\newtheorem{xca}[section]{Exercise}
\newtheorem*{ans}{Answer}

\theoremstyle{remark}
\newtheorem{remark}[theorem]{Remark}

\numberwithin{equation}{section}

\begin{document}

%%%%%%%%%%%%%%%%%%%%%%%%%%%%%%%%%%%%%%%%%%%%%%%%%%%%%%%%%%%%%%%%%%%%%%%%%%%%%%
% Title Page
%%%%%%%%%%%%%%%%%%%%%%%%%%%%%%%%%%%%%%%%%%%%%%%%%%%%%%%%%%%%%%%%%%%%%%%%%%%%%%

\def\thetitle{On The Rank of a Polynomial Matrix}
\def\theauthor{Patrick McLaren}
\def\theemail{patrick.mclaren001@umb.edu}

\begin{titlepage}
  \begin{center}
    \textsc{\LARGE \thetitle}\\[0.75cm]
    \textsc{By}\\[0.25cm]
    \textsc{\large \theauthor}\\[1cm]
  \end{center}
\end{titlepage}

%%%%%%%%%%%%%%%%%%%%%%%%%%%%%%%%%%%%%%%%%%%%%%%%%%%%%%%%%%%%%%%%%%%%%%%%%%%%%%
% Copyright Page
%%%%%%%%%%%%%%%%%%%%%%%%%%%%%%%%%%%%%%%%%%%%%%%%%%%%%%%%%%%%%%%%%%%%%%%%%%%%%%

\thispagestyle{empty}

\null
\vfill

\textsc{\copyright \! 2014 Patrick McLaren \\ All Rights Reserved}

\newpage

%%%%%%%%%%%%%%%%%%%%%%%%%%%%%%%%%%%%%%%%%%%%%%%%%%%%%%%%%%%%%%%%%%%%%%%%%%%%%%
% Abstract
%%%%%%%%%%%%%%%%%%%%%%%%%%%%%%%%%%%%%%%%%%%%%%%%%%%%%%%%%%%%%%%%%%%%%%%%%%%%%%

\thispagestyle{empty}

\section*{Abstract}

\begin{center}
  \begin{minipage}{0.6\textwidth}
    \begin{flushleft}
      This is where the abstract goes
    \end{flushleft}
  \end{minipage}
\end{center}

\newpage

%%%%%%%%%%%%%%%%%%%%%%%%%%%%%%%%%%%%%%%%%%%%%%%%%%%%%%%%%%%%%%%%%%%%%%%%%%%%%%
% Main Content
%%%%%%%%%%%%%%%%%%%%%%%%%%%%%%%%%%%%%%%%%%%%%%%%%%%%%%%%%%%%%%%%%%%%%%%%%%%%%%

\setcounter{page}{1}

\section{Preliminaries}

\subsection{Review of Linear Algebra}

\subsubsection{Vector Spaces}

\begin{definition}
    Let $F$ be a field. Let $V$ be a subset of $F$, and suppose that $V$
    satisfies the following properties
    \begin{enumerate}[i]
        \item $v_1, v_2 \in V \implies v_1 + v_2 \in V$
        \item $v_1 \in V, c \in F \implies c * v_1 \in V$
    \end{enumerate}
    In this case, $V$ is said to be a \emph{vector space} over $F$.
\end{definition}

\subsubsection{Matrix Representation of a Linear Transformation}

\subsubsection{Row Echelon Form}

The setting in which one wants to produce an upper triangular matrix usually involves some desire to solve the equation
    \begin{equation*}
      AX = B
    \end{equation*}
    where $A$ is an invertible linear transformation, and $X, B$ are vectors (or elements of a free module). One can perform \emph{Gaussian Elimination} to produce an upper triangular matrix (or \emph{Gauss-Jordan Elimination} to produce the reduced row echelon form of a matrix) by applying a sequence of elementary row operations to the system. \\

    Now, these elementary row operations can be represented as elementary matrices, say $E_1, \ldots, E_k$. Applied to the original system, we have
    \begin{align*}
      E_k \cdots E_1 AX &= E_k \cdots E_1 B\\
      A'X &= B'
    \end{align*}

    We know that these operations do not change the solution to the equation by application of the cancellation law, since elementary matrices are invertible. Then, one may obtain $X$ by back-substitution. If $A$ was not invertible, then a pseudoinverse may be obtained to describe all solutions that satisfy the equation.

\section{Rings}

\subsection{Definitions}

\subsection{Polynomial Rings}

\subsection{Ideals}

\begin{definition}
  Let $R[X]$ be a polynomial ring, and let $I \subset R$. Suppose that $I$ satisfies the following properties:
  \begin{enumerate}[i]
  \item $0 \in I$
  \item If $f, g \in I$, then $f + g \in I$
  \item If $f \in I, h \in R$, then $hf \in I$
  \end{enumerate}
  In this case, $I$ is said to be an \emph{ideal}.
\end{definition}

\begin{definition}
  Let $R[X]$ be as above, and let $f_1, \ldots, f_n$ be polynomials in $R[X]$. We define
  \begin{equation*}
    \langle f_1, \ldots f_n \rangle = \left\{ \sum_{i = 1}^n h_i f_i \mid h_1, \ldots, h_n \in R[X] \right\}
  \end{equation*}
\end{definition}

Given polynomials $f_1, \ldots, f_n$, it is easy to see that $\langle f_1, \ldots, f_n \rangle$ is an ideal. Indeed, $h_1 = h_2 = \cdots = h_n = 0$, then $\sum h_i f_i = 0$, so $0 \in \langle f_1, \ldots, f_n \rangle$. For $f_j, f_k \in \{ f_1, \ldots, f_n \}$, set $h_i$ equal to $1$ if $i = j$ or $k$, and $0$ otherwise, so $f_j + f_k \in \langle f_1, \ldots, f_n \rangle$. For $f_k \in \{ f_1, \ldots, f_n \}$, set $h_i = c$ if $i = k$, and $0$ otherwise, so $c f_k \in \langle f_1, \ldots, f_n \rangle$. So, $\langle f_1, \ldots, f_n \rangle$ is an ideal in $R[X]$, namely, the \emph{ideal generated by $f_1, \ldots, f_n$}.

\leavevmode

\todo[inline]{Hilbert Basis Theorem}

\leavevmode

\todo[inline]{Noetherian Rings}

\subsection{Gr\"oebner Bases}

\todo{Paraphrase this - copied from Cox, Little \& O'Shea}
\begin{definition}
  Fix a monomial order. A finite subset $G = \{ g_1, \ldots, g_t \}$ of an ideal $I$ is said to be a \emph{Gr\"oebner basis} if
  \begin{equation*}
    \langle \mathrm{LT}(g_1), \ldots, \mathrm{LT(g_n)} \rangle = \langle \mathrm{LT}(I) \rangle
  \end{equation*}
\end{definition}

\todo[inline]{Uniqueness of a reduced Gr\"oebner basis}

\subsection{Localization of a Ring}

\leavevmode \\

\todo[inline, caption={Notes on Section}, color=green!40]{
  \begin{minipage}{1.0\linewidth}
    I'm adding this section in case my method for clearing pivot columns is incorrect. Of course, the duty to provide evidence for it's truth should lie with me - I'm hopeful that researching this section will provide me with some hints. \\

    My main source of uncertainty is in the case that we have to move to the field of fractions in order to continue row reduction.
  \end{minipage}
}

\subsubsection{McCoy's Theorem \& Zero Divisors}

\section{Modules}

\subsection{Definitions}

\subsection{Module over a Polynomial Ring}

\subsection{Row Echelon Form of a Polynomial Matrix}

\begin{lstlisting}[caption=Computing the Row Echelon Form of a Polynomial Matrix]
    for i in (1 ... min(row_size, column_size)):
        target_lcm = lcm(matrix[i][i], matrix[j][i])

        pivot_quotient = target_lcm / matrix[i][i]
        clear_quotient = target_lcm / matrix[j][i]

        matrix.scala_mult(clear_quotient, j)
        matrix.subtract_row(pivot_quotient, i, j)
\end{lstlisting}

\section{When does a Polynomial Matrix have Rank $\leq$ n?}

\section{Examples}

\section{Conclusion}

\section*{Appendix}

\subsection*{Source Code}

\leavevmode \\

\todo[inline]{Include relevant code without inline documentation from https://github.com/patrickmclaren/math-thesis}

\subsection*{Code Documentation}

\leavevmode \\

\todo[inline]{Include documentation (ensure short and unstyled) from https://github.com/patrickmclaren/polyrank-docs}

\begin{thebibliography}{99}
  \bibitem{cox-little-oshea}D. Cox, J. Little, and D. O'Shea, {\em Ideals, Varieties, and Algorithms: An Introduction to Computational Algebraic Geometry and Commutative Algebra}
\end{thebibliography}

\end{document}